\documentclass{lab}

\renewcommand{\AA}{\ensuremath{\mathring{A}}}

\begin{document}

\labtitle{5.7.1}{Измерение углового распределения жесткой компоненты космического излучения}{5~сентября~2019~г.}{12~сентября~2019~г.}

\section*{Постановка эксперимента}

\begin{quote}
	\textbf{{\normalsize Цель работы: }}
	с помощью телескопа из двух сцинтилляторов измерить угловое распределение жесткой компоненты космического излучения и на основе полученных данных оценить время жизни мюона.
\end{quote}

\begin{quote}
	\textbf{{\normalsize Оборудование: }}
	телескоп с двумя сцинтилляторами в режиме синхронизации, счетчик с секундомером.
\end{quote}

\section*{Теория}

$\widetilde{U}$

В работе рассматривается вторичное космическое излучение, полученное путем воздействия первичного космического излучение на атмосферу Земли. Первичное излучение в основном состоит из протонов $ (\approx 90\%) $, $ \alpha $-частиц $ (\approx 10\%) $ и небольшого количества ядер более тяжелых элементов. Вторичная компонента создается при взаимодействии быстрых частиц с ядрами атомов воздуха, а вблизи уровня моря происходит в основном уменьшение интенсивности космического излучения из-за постепенного поглащения частиц в воздухе.\\

Основная часть мюонов рождается в верхних слоях атмосферы. Высота атмосферы до слоя, где рождаются мюоны, равна $ L_0 \approx 15~км $, а также учитывая, что радиус земли $ R_0 \approx 6400~км $, можем представить поверхность Земли плоскостью в пределах $ 0 \leqslant \Theta \leqslant 75^{\circ} $ от вертикали (погрешность меньше $ 1\% $).\\

Учитывая вероятности попадания в детектор мюонов из верхних слоев и лавинно-образованных вблизи, получаем для длины распадного пробега $ L = v\tau $:
\begin{equation}\label{eq-1}
L = \beta c \dfrac{\tau_0}{\sqrt{1 - \beta^2}} = \beta c \tau_0 \dfrac{E_{\mu}}{m_{\mu} c^2},
\end{equation}

где $ v = \beta c $ -- скорость мюона, $ \tau $ -- время жизни движущегося мюона, $ \tau_0 $ -- время жизни покоящегося мюона, $ E_{\mu} = m_{\mu} c^2 / \sqrt{1 - \beta^2} \approx 4 \cdot 10^9~эВ $ -- полная энергия мюона, $ m_{\mu} = 105.8~МэВ / c^2 $. А также подчеркнем, что $ \beta \approx 1 $.\\

Отношение числа мюонов, идущих под зенитным углом $ \Theta $, к числу вертикально падающих мюонов можно записать через произведение вероятностей прохождения пути $ L $ без поглащения и без распада в виде:
\begin{equation}\label{eq-2}
\dfrac{N(\Theta)}{N(0^{\circ})} = \dfrac{P_1(\Theta)}{P_1(0^{\circ})} \dfrac{P_2(\Theta)}{P_2(0^{\circ})} = (\cos (\Theta))^n \dfrac{\exp(-L(\Theta)/L)}{\exp(-L_0/L)},
\end{equation}

где $ P_1(\Theta), P_2(\Theta) $ -- те самые вероятности, $ n $ -- показатель зависимости $ I = I_0 (\cos (\Theta))^n $. Расстояние $ L(\Theta) = L_0 / \cos (\Theta) $ -- вследствие приближения плоскостью поверхность Земли.

\newpage

\section*{Выполнение работы}

\begin{enumerate}
\item 
Получим зависимость количества пойманных мюонов $ N $ от зенитного угла установки $ \Theta $ за $ t = 200~с $.

\begin{table}[H]
	\centering
	\begin{tabular}{|c|cccccccccc|}
		\hline
		$ N_1,~шт $          & 121 & 124 & 124 & 107 & 103 & 97  & 79  & 59  & 47  & 46  \\
		$ N_2,~шт $          & 111 & 120 & 117 & 109 & 96  & 87  & 81  & 68  & 43  & 57  \\
		$ N_3,~шт $          & 139 & 121 & 113 & 111 & 101 & 83  & 76  & 60  & 44  & 49  \\
		$ N,~шт $            & 124 & 122 & 118 & 109 & 100 & 89  & 79  & 62  & 45  & 51  \\ \hline
		$ \Theta,~^{\circ} $ & 0   & 10  & 20  & 30  & 40  & 50  & 60  & 70  & 80  & 90  \\ 
		$ I(\Theta),~с^{-1} $& 0.62& 0.61& 0.59& 0.55& 0.50& 0.45& 0.39& 0.31& 0.22& 0.25\\ \hline
	\end{tabular}
	\caption{Интенсивность $ I(\Theta) $ попадания мюонов в детектор за $ t = 200~с $. $ N = (N_1 + N_2 + N_3) / 3 $; \ \ \ \ \ $ I(\Theta) = N(\Theta) / t $.}
	\label{tab-1}
\end{table}

\item 
Критические случаи углов не будем принимать во внимание из-за неопределенностей и неточностей. Построим график $ \ln (I)~от~\ln (\cos (\Theta)) $ для нахождения~$ n $.

\begin{figure}[H]
	\centering
	\begin{tikzpicture}
	
	\pgfplotstableread{
		 X	    Y		x-err	y-err
		 0.00	-0.48		0.05	0.05
		-0.02	-0.50		0.05	0.05 
		-0.06	-0.53		0.05	0.05
		-0.14	-0.61		0.05	0.05
		-0.27	-0.69		0.05	0.05
		-0.44	-0.81		0.05	0.05
		-0.69	-0.93		0.05	0.05
		-1.07	-1.17		0.05	0.05
		-1.75	-1.50		0.05	0.05
	}{\mytable}
	
	\begin{axis}[
	width = 0.8\textwidth,
	grid = major,
	xlabel = $ \ln (\cos (\Theta)) $,
	ylabel = $ \ln (I (\Theta)) $,
	ymin = -1.6,
	ymax = -0.4,
	xmin = -1.8,
	xmax = 0.2
	]
	
	\addplot[
	only marks,
	color = red,
	mark = *,
	error bars/.cd,
	x dir = both,
	x explicit,
	y dir = both,
	y explicit
	]
	table[
	x error = x-err,
	y error = y-err
	] {\mytable};
	
	\addplot[
	mark = none,
	color = red
	]
	table[
	y = {create col/linear regression={y=Y}}
	] % compute a linear regression from the
	{\mytable};
	
	\end{axis}
	\end{tikzpicture}
	\caption{Нахождение коэффициента $ n $ по коэффициенту наклона прямой.}
	\label{g_1}
\end{figure}

\item 

Из графика (рис. \ref{g_1}) по данным из таблицы \ref{tab-1} находим $ n \approx 1.7 $, тогда как в теории $ n_T \approx 1.6 $. Следовательно коэффициент найден достаточно точно.

\newpage

\item 
Воспользовавшись формулами \eqref{eq-1} и \eqref{eq-2}, а также приняв во внимание $ \beta~\approx~1 $, $ L_0~\approx~15~км $, $ n~=~1.7 $, получаем формулу для расчета времени жизни мюона:
\begin{equation}\label{eq-3}
\tau_0 = \dfrac{[L_0 - L(\Theta)] \cdot m_{\mu} c}{E_{\mu} \ln \left[\dfrac{N(\Theta)}{N(0^{\circ})} \dfrac{1}{(\cos (\Theta))^n}\right]}
\end{equation}

\item 
В результате получим для каждого зенитного угла $ \Theta $ определенное значение $ \tau_0 $.

\begin{table}[H]
	\centering
	\begin{tabular}{|c|ccccccc|}
		\hline
		$ \tau_0 \cdot 10^{-6},~с $ & 2.1 & 1.4 & 1.7 & 1.7 & 1.7 & 1.8 & 2.2 \\
		$ \Theta,~^{\circ} $        & 10  & 20  & 30  & 40  & 50  & 60  & 70  \\ \hline
	\end{tabular}
	\caption{$ \tau_0 $ для адекватных зенитных углов $ \Theta $.}
	\label{tab-2}
\end{table}

\item 
Среднее значение времени жизни мюона $ \overline{\tau_0} = 1.8 \cdot 10^{-6}~с $.

\end{enumerate}

\subsection*{Итоги}

Провели исследование жесткого космического излучения на уровне моря, а также проверили некоторые теоретические формулы.\\

Провели оценку времени жизни мюона в лабораторных условиях: $ \overline{\tau_0} = 1.8 \cdot 10^{-6}~с $. Табличное значение: $ {\tau_T} = 2.2 \cdot 10^{-6}~с $. Следовательно, оценка времени жизни мюона проведена успешно.

\end{document}