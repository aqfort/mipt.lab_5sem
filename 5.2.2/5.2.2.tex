\documentclass{lab}

\renewcommand{\AA}{\ensuremath{\mathring{A}}}

\begin{document}

\labtitle{5.2.2}{Изучение спектров водорода и йода}{14~ноября~2019~г.}{21~ноября~2019~г.}

\section*{Постановка эксперимента}

\begin{quote}
	\textbf{{\normalsize Цель работы: }}откалибровать спектрометр по линиям неона и ртути, определить координаты линий серии Бальмера атомарного водорода, рассчитать постоянную Ридберга. Вычислить энергию колебательного кванта молекулы йода и энергию ее диссоциации в основном и возбужденном состоянии.
\end{quote}
	
\section*{Экспериментальная установка}

Длины волн спектральных линий водородоподобного атома:
\begin{equation}
\dfrac{1}{\lambda_{nm}} = RZ^2 \left( \dfrac{1}{n^2} - \dfrac{1}{m^2} \right)
\end{equation}
Энергия кванта в возбужденном состоянии:
\begin{equation}
h\nu_2 = \dfrac{h(\nu_{1.5} - \nu_{1.0})}{5}
\end{equation}
Энергия диссоциации:
\begin{equation}
D_1 = h\nu_{гр} - E_A
\end{equation}

\section*{Выполнение работы}

\begin{enumerate}

\item
Калибровка барабана спектрометра линиями неона и ртути:
\begin{table}[H]
	\centering
	\begin{tabular}{|cc|cccccc|}
	\hline
	\multirow{2}{2cm}{Неон}  & $\lambda,~\AA{}$ & 7032 & 6402 & 6164 & 6030 & 5852 & 5401 \\
	                         & $\varphi,~^{\circ}$ & 2586 & 2370 & 2270 & 2212 & 2128 & 1870 \\ \hline
	\multirow{2}{2cm}{Ртуть} & $\lambda,~нм$    & 623.4& 579.1& 577  & 546.1& 491.6& 404.7\\
	                         & $\varphi,~^{\circ}$ & 2302 & 2094 & 2084 & 1902 & 1486 & 822  \\
	\hline
		
	\end{tabular}
	\label{tab1}
\end{table}

\item 
Определение длин волн водорода с помощью калибровочного графика (графики приведены ниже) и проверка формулы сериальной закономерности:
\begin{table}[H]
	\centering
	\begin{tabular}{|ccccc|}
		\hline
		$\varphi,~^{\circ}$ & $\lambda,~нм$ & n & $1/\lambda,~нм^2$ & $1/4 - 1/n^2$ \\ \hline
		114  & 393 & 6 & 0.00253 & 0.2222 \\
		530  & 401 & 5 & 0.00249 & 0.2100 \\
		1166 & 458 & 4 & 0.00218 & 0.1875 \\
		2160 & 662 & 3 & 0.00151 & 0.1389 \\
		\hline
		
	\end{tabular}
	\label{tab2}
\end{table}

\item 
Из зависимости $ 1/\lambda $ от $ 1/4 - 1/n^2 $ $(коэффициент~наклона \cdot 10^4~см^{-1})$ находим постоянную Ридберга: $ R = (11.5 \pm 0.2) \cdot 10^4 ~ см^{-1} $.

\item
Приведем графики градуировок:
\begin{figure}[H]
	\centering
	\begin{tikzpicture}
	
	\pgfplotstableread{
		X	    Y			x-err		y-err
		2586	7032		20	 		100
		2370	6402		20			100
		2270	6164		20			100
		2212	6030		20			100
		2128	5852		20			100
		1870	5401		20			100
	}{\mytable}
	
	\begin{axis}[
	width = 0.75\textwidth,
	grid = major,
	xlabel = $ \varphi\text{,}~^{\circ} $,
	ylabel = $ \lambda\text{,}~\AA{} $,
	ymin = 5200,
	ymax = 7200,
	xmin = 1800,
	xmax = 2700
	]
	
	
	\addplot[
	only marks,
	color = red,
	mark = *,
	error bars/.cd,
	x dir = both,
	x explicit,
	y dir = both,
	y explicit
	]
	table[
	x error = x-err,
	y error = y-err
	] {\mytable};
	
	\addplot[
	mark = none,
	color = red
	]
	table[
	y = {create col/linear regression={y=Y}}
	] % compute a linear regression from the
	{\mytable};
	
	\end{axis}
	\end{tikzpicture}
	\caption{$ y = 0.0012x^2 - 3.0077x + 6883.5 $}
	\label{graph1}
\end{figure}

\begin{figure}[H]
	\centering
	\begin{tikzpicture}
	
	\pgfplotstableread{
		X	    Y			x-err		y-err
		2302	623.4		20	 		10
		2094	579.1		20			10
		2084	577 		20			10
		1902	546.1		20			10
		1486	491.6		20			10
		822 	404.7		20			10
	}{\mytable}
	
	\begin{axis}[
	width = 0.75\textwidth,
	grid = major,
	xlabel = $ \varphi\text{,}~^{\circ} $,
	ylabel = $ \lambda\text{,}~\text{нм} $,
	ymin = 350,
	ymax = 650,
	xmin = 600,
	xmax = 2400
	]
	
	
	\addplot[
	only marks,
	color = red,
	mark = *,
	error bars/.cd,
	x dir = both,
	x explicit,
	y dir = both,
	y explicit
	]
	table[
	x error = x-err,
	y error = y-err
	] {\mytable};
	
	\addplot[
	mark = none,
	color = red
	]
	table[
	y = {create col/linear regression={y=Y}}
	] % compute a linear regression from the
	{\mytable};
	
	\end{axis}
	\end{tikzpicture}
	\caption{$ y = 5\cdot10^{-5}x^2 + 0.0003x + 375.75 $}
	\label{graph2}
\end{figure}

\item 
Используя калибровку, определим длины волн в спектре йода:
\begin{table}[H]
	\centering
	\begin{tabular}{|ccc|}
		\hline
		$\varphi,~^{\circ}$ & $\lambda,~\AA{}$ & $h\nu$ \\ \hline
		2054  & 6264 & $ h\nu_{1.0} = 1.979 $ \\
		1970  & 6060 & $ h\nu_{1.5} = 2.046 $ \\
		1356  & 5340 & $ h\nu_{гр} = 2.322 $  \\
		\hline
		
	\end{tabular}
	\label{tab3}
\end{table}

\item 
Теперь расчеты:
\begin{gather*}
	h\nu_2 = (h\nu_{1.5} - h\nu_{1.0})/5 = 0.0133 \pm 0.0003~эВ\\
	h\nu_{1.0} = (E_2 - E_1) + h\nu_2 \left( 1 + \dfrac{1}{2} \right) - \dfrac{3}{2} h\nu_1\\
	h\nu_{1.0} = h\nu_{el} + \dfrac{3}{2} (h\nu_2 - h\nu_1) \then h\nu_{el} = 2.02 \pm 0.04~эВ\\
	h\nu_{гр} = h\nu_{el} + D_2 = D_1 + E_A
\end{gather*}
Энергия диссоциации в возбужденном состоянии: $ D_2 = 0.30 \pm 0.01~эВ $.
Энергия диссоциации в основном состоянии: $ D_1 = 1.08 \pm 0.04~эВ $.

\end{enumerate}

\subsection*{Итоги}

Водородные длины волн соответствуют сериальной закономерности. Найдена постоянная Ридберга: $ R = (11.5 \pm 0.2) \cdot 10^4 ~ см^{-1} $. Табличное значение: $ R = 109677.5~см^{-1} $. Для йода найдены: энергия электронного перехода $ h\nu_{el} = 2.02 \pm 0.04~эВ $, энергии диссоциации в основном состоянии $ D_1 = 1.08 \pm 0.04~эВ $ и возбужденном $ D_2 = 0.30 \pm 0.01~эВ $.

\end{document}